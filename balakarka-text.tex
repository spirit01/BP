%%%%%%%%%%%%%%%%%%%%%%%%%%%%%%%%%%%%%%%%%%%%%%%%%%%%%%%%%%%%%%%%%%%%
%% I, the copyright holder of this work, release this work into the
%% public domain. This applies worldwide. In some countries this may
%% not be legally possible; if so: I grant anyone the right to use
%% this work for any purpose, without any conditions, unless such
%% conditions are required by law.
%%%%%%%%%%%%%%%%%%%%%%%%%%%%%%%%%%%%%%%%%%%%%%%%%%%%%%%%%%%%%%%%%%%%

\documentclass[
  digital, %% This option enables the default options for the
           %% digital version of a document. Replace with `printed`
           %% to enable the default options for the printed version
           %% of a document.
  table,   %% Causes the coloring of tables. Replace with `notable`
           %% to restore plain tables.
  lof,     %% Prints the List of Figures. Replace with `nolof` to
           %% hide the List of Figures.
  lot,     %% Prints the List of Tables. Replace with `nolot` to
           %% hide the List of Tables.
  %% More options are listed in the user guide at
  %% <http://mirrors.ctan.org/macros/latex/contrib/fithesis/guide/mu/sci.pdf>.
]{fithesis3}
%% The following section sets up the locales used in the thesis.
\usepackage[resetfonts]{cmap} %% We need to load the T2A font encoding
\usepackage[T1,T2A]{fontenc}  %% to use the Cyrillic fonts with Russian texts.
\usepackage[
  main=czech, %% By using `czech` or `slovak` as the main locale
                %% instead of `english`, you can typeset the thesis
                %% in either Czech or Slovak, respectively.
  german, russian, czech, slovak %% The additional keys allow
]{babel}        %% foreign texts to be typeset as follows:
%%
%%   \begin{otherlanguage}{german}  ... \end{otherlanguage}
%%   \begin{otherlanguage}{russian} ... \end{otherlanguage}
%%   \begin{otherlanguage}{czech}   ... \end{otherlanguage}
%%   \begin{otherlanguage}{slovak}  ... \end{otherlanguage}
%%
%% For non-Latin scripts, it may be necessary to load additional
%% fonts:
\usepackage{paratype}
\def\textrussian#1{{\usefont{T2A}{PTSerif-TLF}{m}{rm}#1}}
%%
%% The following section sets up the metadata of the thesis.
\thesissetup{
    date          = \the\year/\the\month/\the\day,
    university    = mu,
    faculty       = sci,
    department    = Ústav chemie,
    departmentEn  = Department of Mathematics and
                    Statistics,
    programme     = Chemie,
    programmeEn   = Chemistry,
    field         = Fyzikální chemie,
    fieldEn       = Physical chemistry,
    type          = bc,
    author        = Petra Hrozková,
    gender        = f,
    advisor       = doc. Mgr. M. Munzarová Dr. rer. nat ,
    title         = Studium vlivu koordinačního prostředí atomů Si a P na tvorbu SiO6 center metodami EHT a DFT,
    TeXtitle      = Studium vlivu koordinačního prostředí atomů Si a P na tvorbu SiO6 center metodami EHT a~DFT,
    titleEn       = A combined EHT/DFT study of Si and P coordination environment influence on the creation of SiO6 centers,
    TeXtitleEn    = The Principles of the Typesetting of
                    Mathematics in \TeX: the Program,
    keywords      = {klíčové slovo 1, klíčové slovo 2, ...},
    TeXkeywords   = {klíčové slovo 1, klíčové slovo 2, \ldots},
    keywordsEn    = {keyword1, keyword2, ...},
    TeXkeywordsEn = {keyword1, keyword2, \ldots},
}
\thesislong{abstract}{
    This is the abstract of my thesis, which can

    span multiple paragraphs.
}
\thesislong{abstractEn}{
    This is the English abstract of my thesis, which can

    span multiple paragraphs.
}
\thesislong{thanks}{
    This is the acknowledgement for my thesis, which can

    span multiple paragraphs.
}
%% The following section sets up the bibliography.
\usepackage{csquotes}
\usepackage[              %% When typesetting the bibliography, the
  backend=biber,          %% `numeric` style will be used for the
  style=numeric,          %% entries and the `numeric-comp` style
  citestyle=numeric-comp, %% for the references to the entries. The
  sorting=none,           %% entries will be sorted in cite order.
  sortlocale=auto         %% For more unformation about the available
]{biblatex}               %% `style`s and `citestyles`, see:
%% <http://mirrors.ctan.org/macros/latex/contrib/biblatex/doc/biblatex.pdf>.
\addbibresource{example.bib} %% The bibliograpic database within
                          %% the file `example.bib` will be used.
\usepackage{makeidx}      %% The `makeidx` package contains
\makeindex                %% helper commands for index typesetting.
%% These additional packages are used within the document:
\usepackage{paralist}
\usepackage{amsmath}
\usepackage{amsthm}
\usepackage{amsfonts}
\usepackage{url}
\usepackage{menukeys}
\begin{document}
\chapter{Úvod}
Theses are rumoured to be the capstones of education, so I decided
to write one of my own. If all goes well, I will soon have a
diploma under my belt. Wish me luck!

\begin{otherlanguage}{czech}
Říká se, že závěrečné práce jsou vyvrcholením studia a tak jsem se
rozhodl jednu také napsat. Pokud vše půjde podle plánu, odnesu si
na konci semestru diplom. Držte mi palce!
\end{otherlanguage}



\chapter{Teoretická část}
Znalosti o chemické vazbě jsou klíčovou znalostí pro každý experiment. S touto znalostí je možno předpokládat průběh experimentu a částečně i strukturu. Tato znalost v historii nebyla samozřejmostí. S chemickou vazbou jsou neoddělitelně spojeny atomy. Počátky atomové teorie datujeme v letech 430-310 př.n.l v Řecku. Autorem je Démokritos, který předpokládal, že hmota je stvořena z malých, dále nedělitelných částí. Počátky molekulové struktury jsou v 18.století, kdy John Dalton použil atomovou teorii k vysvětlení chemických reakcí. Zároveň formuloval zákon stálých poměrl slučovacích a zákon stálých poměrů násobných. V roce 1852 se objevuje myšlenka valence elektronů, která ovšem není kompletní do objevení elektronu 1897 Thomsonem. \\
Lewis jako první nahlížel na chemickou vazbu jako sdílení elektronů. \cite{Munzarova1996thesis} \\

Molekulové orbitaly - přechod od atomů k molekule je kvalitativní změna, kdy elektrony jsou vystaveny působení více jader a dochází ke ztrátě kulové symetrie.  \cite{polak2000obecna}

\section{Hybridizace atomových orbitalů}
Hybridizace je myšlenkový postup,  který usnadňuje pochopení struktury atomů. 
\section{Hypervalence}
\section{Teorie molekulových orbitalů}

\section{Chemické výpočetní metody}
Metody výpočetní chemie lze ve stručnosti shrnutou na \textit{Ab inito} metody, semi-empirické metody a DFT metody.

\section{Density Functional Theorem - DFT}
DFT metody jsou založeny na vztahu mezi elektronovou hustotou a celkovou energii systému, teorie funkcionální hustoty.  Tento výpočetní model byl objeven už v roce 1920, pro chemii začal mít význam až v 60. letech 20. století. Výpočetně porobně náročná jako HF, ale přesnější, obsahují korelční energii.\cite{lechamolecularmodeling}
V roce 1964 uveřejnili Koch a Hohenberg dva teorémy.
\subsection{První Hohenberg- Kohn teorém}

\subsection{Druhý Hohenberg-Kohn teorém}
Druhý teorém je postaven na variačním principu.
\cite{koch2000chemist} 

\chapter{Praktická část}


{\csname captions\languagename\endcsname %% Temporarily override
%% the BibLaTeX localization with the original babel definitions.
\makeatletter %% Use the correct localization of the quotations.
  \thesis@selectLocale{\thesis@locale}\makeatother
\printbibliography[heading=bibintoc]} %% Print the bibliography.
\appendix %% Start the appendices.
\chapter{An appendix}
Here you can insert the appendices of your thesis.




\end{document}
